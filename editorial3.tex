\section{最近点対の距離}
最近点対の距離が$k$以上になる確率を$P(k)$とおく.
[0,1]から最近点対の距離が$k$以上になるように点を$N$個選ぶ操作は,[0, 1 - (N-1)k]から点を$N$個選ぶ操作と対応する(各区間から$k$だけ抜くことを考えれば良い).よって
\[
P(k) = (1-(N-1)k)^N
\]

最近点対の距離を$M$とすると$P$は$M$の累積密度関数に相当する.よって$P$を微分すれば$M$の確率密度関数$p$がもとまる.ここで$P(k)$ が$M$が$k$ \textbf{以上}となる確率であることに注意しなければならない.つまり,符号をひっくり返す必要がある.
\[
p(k) = - \frac{\mathrm{d}P}{\mathrm{d}k}(k)
\]


求めたい値は$\int_0^{\frac{1}{N-1}} k p(k) \mathrm{d}k$なので,あとは計算するだけ.部分積分をすると便利.
\begin{align*}
  &\int_0^{\frac{1}{N-1}} k p(k) \mathrm{d}k\\
  &= \left [-k P(k)\right]_0^{\frac{1}{N-1}} + \int_0^{\frac{1}{N-1}} P(k) \mathrm{d}k\\
  &= \int_0^{\frac{1}{N-1}} P(k) \mathrm{d}k\\
  &= \left[ \frac{(1-(N-1)x)^{N+1}}{1-N^2} \right]_0^\frac{1}{N-1} \\
  &= \frac{1}{N^2-1}
\end{align*}
