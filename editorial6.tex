\section{平均より長い点対の個数}
最小値と最大値を$L,R$とする.
$R,L$が決まっているきの条件付き期待値を考える.スケールが違うだけなので,平均より大きいものの個数は条件なしの期待値と等しい.
よって,$L=0,R=1$の場合のみ考えれば良い.
これは区間を$N-2$箇所で切って,できる$N-1$個の区間のうち長さが$\frac{1}{N-2}$以上のものの個数である.
$l = \frac{1}{N-2}$とする.
一番左の区間が長さ$l$以上になる確率は$(1-l)^{N-2}$である.
ある切る場所について,そこから右に長さ$l$に何もなく,$[0,1]$区間の端が来ることもない確率は$(1-l)^{N-2}$である.
よって$N-2$個の区間についてそれが長さ$l$以上になる確率は$(1-l)^{N-2}$.よってそのような区間の個数の期待値は
$(N-2)(1-l)^{N-2} = (N-2)(1-\frac{1}{N-2})^{N-2}$.

ちなみに$N \to \infty$のとき$\frac{\mathrm{答え}}{N}$は$\frac{\mathrm{e}^{-1}}$に収束する.
