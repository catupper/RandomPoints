\section{最大値}
最大値が$k$以下になる確率を$P(k)$とおく.
$X_i$が$k$以下になる確率は$k$なので,$P(k) = k^N$である.
確率変数$M$ を $M = \max(X_1, \ldots, X_N)$で定義すると,$P(k)$は$M$の累積密度関数に相当する.微分すると確率密度関数$p(k)$がわかる.
\[
p(k) = \frac{\mathrm{d}P}{\mathrm{d}k}(k) = Nk^{N-1}
\]
求めたい値は$\int_0^1 k p(k) \mathrm{d}k$なので,あとは計算するだけ.
\begin{align*}
  &\int_0^1 k p(k) \mathrm{d}k\\
  &= \int_0^1 Nk^N \mathrm{d}k\\
  &= \left [ \frac{N}{N+1}k^{N+1} \right ]_0^1\\
  &= \frac{N}{N+1}
\end{align*}
