\section{最近点対の距離}
$[0,1]$から$N$点取った時の$k$番目に短い点対の大きさの期待値を$f_k(N)$とする.
問3で求めた通り$f_N(1) = \dfrac{1}{N^2-1}$である.\\
$k$番目に短い点対の大きさの期待値は,$N-1$個の区間から一番短い点対の大きさを引くことを考えると,
\begin{align*}
  &f_k(N)\\
  &= f_1(N) + (1 - (N-1)f_1(N))f_{k-1}(N-1)\\
  &= \frac{1}{N^2-1} + \frac{N}{N+1}f_{k-1}(N-1)
\end{align*}

具体的に計算してみると.
\begin{align*}
  & f_1(N) &=\ & \frac{1}{N^2-1} &\\
  & &=&\  \frac{1}{N+1}\frac{1}{N-1} &\\
  & f_2(N) &=&\  \frac{1}{N^2-1} + \frac{N}{N+1}f_1(N-1) &\\
  &  &=&\  \frac{1}{N^2-1} + \frac{N}{N+1}\frac{1}{N}\frac{1}{N-2}&\\
  &  &=&\  \frac{1}{N^2-1} + \frac{1}{(N+1)(N-2)}&\\
  &  &=&\  \frac{1}{N+1}\left(\frac{1}{N-1} + \frac{1}{N-2}\right)&\\
  & f_3(N) &=&\  \frac{1}{N^2-1} + \frac{N}{N+1}f_2(N-1)&\\
  &  &=&\  \frac{1}{N^2-1} + \frac{N}{N+1}\frac{1}{N}\left(\frac{1}{N-2} + \frac{1}{N-3}\right)&\\
  &  &=&\  \frac{1}{N+1}\left(\frac{1}{N-1} + \frac{1}{N-2} + \frac{1}{N-3} \right)&\\
  & &\vdots&&\\
  & f_k(N) &=&\  \frac{1}{N+1} \left( \frac{1}{N-1} + \frac{1}{N-2} + \cdots + \frac{1}{N-k} \right)&
\end{align*}

